\documentclass{report}
\usepackage[T1]{fontenc}
\usepackage[top=1in, bottom=1in, right=1in, left=1in]{geometry}

\begin{document}
	\fontfamily{lmtt}\selectfont
    \begin{flushleft}
    \section*{Introduction}
    \vspace{-0.1cm}\hrule\vspace{0.2cm}
    \par{Flowcell sensors have many applications; disease detection, refractive index measuring, and reactivity measuring to name a few. These sensors have been operating on the basis of electromagnetic surface phenomena for decades. Most flowcells on the market work by exploiting surface plasma oscillations (SPOs). These oscillations are highly sensitive to changes in the optical properties of the adjacent medium and follow from Maxwell's equations when the dielectric functions of each medium satisfies
    \[
        \frac{\epsilon_{spo}}{\epsilon_{adjacent}} < -1
    \]
		Metals like aluminum, copper, gold, and silver have negative dielectric functions at wavelengths in the red/infrared, so films of these metals are used as to generate SPOs in most flowcell sensors via a process known as Surface Plasmon Resonance (SPR). There are quite a few drawbacks for using metal films, however. Metals are highly reactive so they require regular maintenance. These films also require particular wavelengths of incident light to excite the oscillations. Rather than using metal films, one-dimensional photonic crystals, or multilayers, can be designed to exhibit the phenomenon of surface electromagnetic waves (SEWs) or Bloch surface waves (BSWs), named after the physicist Felix Bloch who was famous for working with periodic systems. These surface waves have the same practical application as SPOs. Multilayers overcome both of the shortcomings of metal films listed here. They can be designed to work for any wavelength and are typically made of nonreactive glass. In addition to these benefits, we expect that our 3-D printed and multilayer-based flowcell sensor will be more sensitive and precise with its measurements and be far cheaper to both build and maintain compared to traditional SPR sensors.}
	\par{To take measurements with our sensor we look at the reflected image of incident laser light. Our multilayer is designed to trap incident light in the last layer at certain angles. This results in a dark band in our reflected image.}
	\end{flushleft}
\end{document}

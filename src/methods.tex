\section*{III. Methods}
\addcontentsline{toc}{section}{III.\hspace{0.14in}Methods}
\hspace{0.25in}
To collect data from our biosensor we first prepare the flowcell chamber with the correct substrate or liquid that acts as our reference point for measuring changes in optical properties. After the chamber is prepared we then turn on the beam and rotate it until the special angle is reached. 

\begin{wrapfigure}{L}{4in}
    \includegraphics[width=4in]{reflectivity_dip.png}
\end{wrapfigure}

As the index of refraction inside the flowcell chamber changes the dark band will translate left or right in our reflected image, depending on whether the index is increasing or decreasing. The shift in location of the band, in pixels, corresponds to an angular shift in the part of the reflected beam giving rise to the dark band. This is shown clearly in ?? as we see that a change from an index of 1.00 (air) to 1.33 (water) corresponds to an angular shift  of about $3^\circ$.

% \begin{wrapfigure}{R}{6cm}
% 	\includegraphics[width=6cm]{darkband.png}
% \end{wrapfigure}

Using the calculated data we can build a model to relate the index of refraction in the chamber to a shift in pixels on our CCD. To construct this relation we require Snell's law and some geometry. Using the classic formula for arclength and the diagram below we find:\\

\begin{figure}[h!]
\begin{center}
\includegraphics[width=5in]{pixelshiftfromangle.png}
\end{center}
\end{figure}

\begin{equation*}
	dx = R d\theta_t
\end{equation*}

Using Snell's law we find

\begin{equation*}
	\theta_t = \arcsin(\sin(n_g \theta_r))
\end{equation*}

Which implies that

\begin{equation*}
\begin{split}
	d\theta_t & = d\left(\arcsin\left(n_g \sin{\theta_r}\right)\right) \\
			  & = \cfrac{n_g \cos{\theta_r}}{\sqrt{1-n_g^2 \sin^{2}(\theta_r)}}\,\,d\theta_r
\end{split}
\end{equation*}

This leaves us with the relation between pixel shift and angular shift:

\begin{equation}
	dx = \cfrac{R n_g \cos{\theta_r}}{\sqrt{1-n_g^2 \sin^{2}(\theta_r)}} \,\,d\theta_r
\end{equation}

Numerically we can determine the relationship between the reflected angle and the index inside the chamber. I wrote a python program to do just that for our multilayer, assuming the chamber is initially filled with water, and a graph of mode position ($\theta_r$) vs index of refraction is plotted below.

\begin{figure}[h]
\begin{center}
\includegraphics[width=5in]{modeposition_vs_index.png}
\caption{The theoretical relation between the angular position of the mode and the index of refraction inside the flowcell chamber. }
\label{fig:modevsindex}
\end{center}
\end{figure}
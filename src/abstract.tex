\section*{ABSTRACT}
\addcontentsline{toc}{section}{\hspace{0.34in}Abstract}
\hspace{0.25in}
In this thesis we present our approach to constructing an ultra low-cost flowcell biosensor. Our sensor is capable of detecting changes in index of refraction on the order of $10^{-3}$, showing that it is well suited for not only index testing but also for surface loading type processes where binding between molecules may occur. Our sensor utilizes 3D printed parts, a one dimensional photonic crystal coupled to a glass prism (rather than a traditional SPR metal-film-prism coupling system), and a CCD to operate. Highly sensitive Bloch Surface Waves (BSWs) can be excited in the outer layer of our multilayer by coupling an incident laser beam to the prism-multilayer structure and the position of the BSWs can be tracked by analyzing the reflected beam. By using this method we are able to bring the cost of manufacturing the sensor to about $\$100$, excluding the photonic crystals which can be fabricated commercially. The largest cost of our system is the CCD used to collect data. \\
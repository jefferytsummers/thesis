\section*{\textbf{V. CONCLUSION}}
\addcontentsline{toc}{section}{V.\hspace{0.18in}Conclusion}
\hspace{0.25in}
In conclusion, this is a fine alternative to traditional flowcell sensors on the market. The parts that need to be purchased are the multilayers, a low power laser, a focusing lens, a CCD (we took one from a \$40 webcam), and a neutral density filter (d=3.5). The rest of the parts can be 3D printed (the CAD files can be found on my GitHub page: https://github.com/jefferytsummers/thesis). This sensor offers many advantages over traditional sensors. The synthesis of 3D printed parts and a photonic crystal provide the capability of operation at any wavelength and any incident angle \cite{farmer2012biosensing}. Users of this sensor can maintain it themselves by cleaning optics, printing off spare parts, and replacing the batteries in the low power laser. The photonic crystal's can be ordered in bulk from a host of different commercial thin film companies.

The sensor does have a few issues, however. The flowcell itself is not quite water tight. This could be solved by either machining the flowcell or coating the chamber with a water tight sealant. Some multilayers may have coupling angles that are awkward to reach with the current design, however the modular nature of the flowcell stage can aid in reaching the coupling angle. In addition to modifying the flowcell stage, a hemispherical prism could be used in place of the right triangular prism used in our aparatus. The largest uncertainty of our device comes from the error in our ability to measure the pixel shift of the surface mode. At the moment a nine by nine gaussian blur is used to smooth out the image of the surface mode leaving us with an uncertainty in the mode position of about $\pm 9$ pixel. This error can be minimized with respect to our measurements if we move the CCD further away. The error would not be reduced but our measurements of pixel shifts would be much larger than they are with the camera at its current distance.
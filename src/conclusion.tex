\section*{\textbf{V. CONCLUSION}}
\addcontentsline{toc}{section}{V.\hspace{0.18in}Conclusion}
\hspace{0.25in}
In conclusion, this is a fine alternative to traditional flowcell sensors on the market. The only parts that need to be bought are the multilayers, a low power laser, a focusing lens, and a CCD. The rest of the parts can be 3D printed. The sensor does have a few issues, however. The flowcell itself is not quite water tight. This could be solved by either machining the flowcell or coating the chamber with a water tight sealant. Some multilayers may have coupling angles that are awkward to reach with the current design, however the modular nature of the flowcell stage can aid in reaching the coupling angle. In addition to modifying the flowcell stage, a hemispherical prism could be used in place of the right triangular prism used in our aparatus. The largest uncertainty of our device comes from the error in our ability to measure the pixel shift of the surface mode. At the moment a nine by nine gaussian blur is used to smooth out the image of the surface mode leaving us with an uncertainty in the mode position of about $\pm 9$ pixel .